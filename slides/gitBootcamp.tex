\documentclass{beamer}

\usetheme{Boadilla}
\usecolortheme{crane}
\usepackage{multicol}
\usepackage{graphicx}

\title{\textbf{Git Bootcamp}}
\subtitle{\textit{Computer Student Society}}
\author{Nate Dolny}
\date{}

\begin{document}

% title page 
\begin{frame}
	\titlepage
\end{frame}

% What is git?
\begin{frame}{What is Git?}
	\frametitle{\textbf{What is Git?}}
	\begin{itemize}
		\item One of the many popular version control systems available
	\end{itemize}

	\begin{center}
		\includegraphics[width=0.2\textwidth]{img/Git-Logo.png} 
		\hspace{1cm}
		\includegraphics[width=0.2\textwidth]{img/Mercurial_logo.png} 
		\hspace{1cm}
		\includegraphics[width=0.2\textwidth]{img/Apache_Subversion_logo.png} 
		\hspace{1cm}
	\end{center}	

	\begin{itemize}
		\item Git is a tool used to manage code but can also be applied to any other 
type of document (\textbf{does not have to be code})
	\end{itemize}
	
	\begin{block}{\textbf{Version Control}}
		\textit{Records changes a person made to a document overtime. It can be used 
by one person or multiple people.}
	\end{block}
\end{frame}

% When should you use Git?
\begin{frame} 
	\frametitle{\textbf{When should you use Git?}}
	\begin{itemize}
		\item \textbf{ALWAYS!!}
		\item Git is your best friend when it comes to any project big or small  
	\end{itemize}
\end{frame}

% Setting up Git
\begin{frame}
	\frametitle{\textbf{Setting up Git}}

	\textbf{Create a Profile}
	\vspace{0.5cm}

	\textit{Set Profile Name}
	\begin{itemize}
		\item git config --global user.name "Peter Parker"
	\end{itemize}

	\textit{Set Profile Email}
	\begin{itemize}
		\item git config --global user.name "Parker.P@midtown.net"
	\end{itemize}

	\textit{Set Default Editor}
	\begin{itemize}
		\item git config --global editor="vim"
	\end{itemize}

	\textit{View our configuration (Optional)}
	\begin{itemize}
		\item git config --global list
	\end{itemize}

	\begin{block}{\textbf{PRO TIP}}
		\textit{Git allows for multiple profiles, so you could have a personal and 
		school profile.}
	\end{block}
\end{frame}

% Command-line Text Editors 
\begin{frame}
	\frametitle{\textbf{Command-line Text Editors}}
	
	\begin{itemize}
		\item A text editor that can run directly in a terminal without a graphical user interface
	\end{itemize}

	\begin{figure}[h]
			\centering
			\includegraphics[width=0.2\textwidth]{img/Vim_logo.png} 
			\hspace{0.5cm}
			\includegraphics[width=0.2\textwidth]{img/nano_logo.png} 
			\hspace{0.5cm}
			\includegraphics[width=0.2\textwidth]{img/Emacs_logo.png} 
	\end{figure}
\end{frame}

% Nano Text Editor 
\begin{frame}
	\frametitle{\textbf{GNU Nano Text Editor}}
		
	\begin{multicols}{2}
		\begin{figure}[h]
			\centering
			\includegraphics[width=0.2\textwidth]{img/nano_logo.png} 
		\end{figure}

		\begin{itemize}
			\item Released in November 18, 1999
			\item User-friendly and easy to learn 
			\item Developed and maintained by volunteers
		\end{itemize}

		\includegraphics[width=0.5\textwidth]{img/nano.png} 
	\end{multicols}

\end{frame}

% Vim Text Editor 
\begin{frame}
	\frametitle{\textbf{Vim Text Editor}}
		
	\begin{multicols}{2}
		\begin{figure}[h]
			\centering
			\includegraphics[width=0.2\textwidth]{img/Vim_logo.png} 
		\end{figure}

		\begin{itemize}
			\item Released in November 2, 1991
			\item Modal editor 
			\item Highly customizable, many plugins available 
		\end{itemize}

		\includegraphics[width=0.5\textwidth]{img/Vim.png} 
	\end{multicols}
\end{frame}

% Emacs Text Editor 
\begin{frame}
	\frametitle{\textbf{Emacs Text Editor}}
		
	\begin{multicols}{2}
		\begin{figure}[h]
			\centering
			\includegraphics[width=0.2\textwidth]{img/Emacs_logo.png} 
		\end{figure}

		\begin{itemize}
			\item Released in March 20, 1985
			\item Created by GNU Project founder Richard Stallman
			\item Highly customizable
			\item Many modes for different purposes such as browsing the web
		\end{itemize}

		\includegraphics[width=0.5\textwidth]{img/Emacs.png} 
	\end{multicols}
\end{frame}

% What makes a good commit
\begin{frame}
	\frametitle{\textbf{What makes a good commit?}}
	
	\begin{itemize}
		\item Detailed messages that someone can read that informs others about the
changes you made to a document. 
	\end{itemize}

	\begin{figure}[h]	
		\centering
		\includegraphics[width=0.5\textwidth]{img/minorchanges.jpeg} 
	\end{figure}

\end{frame}

\begin{frame} 
	\frametitle{\textbf{What makes a good commit?}}
	
	\textbf{A better approach to making commits} 
	\vspace{0.5cm}
	
	\begin{itemize}
		\item \textit{Title:} Add a topic
		\vspace{0.5cm}
		\item \textit{Changes:} Add details about changes made to file(s)
		\vspace{0.5cm}
		\item \textit{State:} Add a brief summary of the state of program or feature
		\vspace{0.5cm}
		\item \textit{TODO:} Make notes about things to fix
	\end{itemize}	

	\begin{block}{\textbf{PRO TIP}}
		Pretend you are sending an email to your friends, group mates, or professor
		informing them of the changes you made to the document.
	\end{block}

\end{frame} 

% Making a Commit 
\begin{frame}
	\frametitle{\textbf{Making a Commit}}

	\textbf{Step 1.} \textit{Add a document to staging}
	\begin{itemize}
		\item git add \(\prec document \succ\)
	\end{itemize}

	
	\vspace{0.5cm}

	\textbf{Step 2.} \textit{View documents that have been staged (Optional)}
	\begin{itemize}
		\item git status
	\end{itemize}

	
	\vspace{0.5cm}

	\textbf{Step 3.} \textit{Remove a document from staging (Optional)}
	\begin{itemize}
		\item git reset \(\prec document \succ\)
	\end{itemize}


	\vspace{0.5cm}

	\textbf{Step 4.} \textit{Record changes to repository}
	\begin{itemize}
		\item git commit
	\end{itemize}


	\vspace{0.5cm}

	\textbf{Step 5.} \textit{Upload changes to remote repository}
	\begin{itemize}
		\item git push 
	\end{itemize}
\end{frame}

% Remote version control platforms 
\begin{frame}
	\frametitle{\textbf{Remote Version Control Platforms}}

	\begin{itemize}
		\item Allows you to store code some place other than your local machine
		\item Allows for multiple people to collaborate on a project 
	\end{itemize}

	\begin{figure}[h]
		\begin{flushleft}
		\includegraphics[width=0.5\textwidth]{img/GitHub_Logo.png} 
		\end{flushleft}
		\centering
		\includegraphics[width=0.5\textwidth]{img/GitLab_logo.png} 
		\begin{flushright}
		\includegraphics[width=0.5\textwidth]{img/Bitbucket_Logo.png} 
		\end{flushright}
	\end{figure}

\end{frame}

% GitHub 
\begin{frame}
	\frametitle{\textbf{GitHub}}
		
	\begin{multicols}{2}
		\begin{figure}[h]
			\centering
			\includegraphics[width=0.3\textwidth]{img/GitHub_logo.png} 
		\end{figure}

		\begin{itemize}
			\item Released in April 2008 
			\item Microsoft acquired Github in 2012 
			\item Over 100 million users
			\item Hosts millions of open source projects
		\end{itemize}

		\includegraphics[width=0.5\textwidth]{img/github_ui.png} 
	\end{multicols}
\end{frame}

% Gitlab
\begin{frame}
	\frametitle{\textbf{GitLab}}
		
	\begin{multicols}{2}
		\begin{figure}[h]
			\centering
			\includegraphics[width=0.3\textwidth]{img/GitLab_logo.png} 
		\end{figure}

		\begin{itemize}
			\item Released in 2011
			\item Over 30 million users
			\item More secure than GitHub
			\item Focuses on collaboration, efficiency, and automation 
			\item Only platform that self-hosting is free 
		\end{itemize}

		\includegraphics[width=0.5\textwidth]{img/gitlab_ui.png} 
	\end{multicols}
\end{frame}

% Intro to OpenSSH Suite
\begin{frame}
	\frametitle{\textbf{Introduction to OpenSSH Suite}}

	\begin{multicols}{2}
		\begin{figure}[h]
			\begin{flushleft}
			\includegraphics[width=0.5\textwidth]{img/openssh.png}
			\end{flushleft}
		\end{figure}
		
		\begin{itemize}
			\item Encrypts all traffic to eliminate eavesdropping, connection hijacking, and other attacks.

			\vspace{0.2cm}
			\item ssh-keygen: Generates, manages, and converts authentication keys for ssh
			\item ssh-add: Adds private keys to the authentication agent
			\item ssh-scp: copies files between hosts on a network 
		\end{itemize}

		\begin{figure}[h]
			\begin{flushright}
			\includegraphics[width=0.5\textwidth]{img/securityMeme.jpeg}
			\end{flushright}
		\end{figure}
	\end{multicols}
\end{frame}

% Create an SSH Key
\begin{frame}
	\frametitle{\textbf{Generate an SSH Key}}
	
	\textbf{Generate an SSH-Key}
	\begin{itemize}
		\item ssh-keygen -t ed25519 -C "Parker.P@midtown.net" 
	\end{itemize}
	\vspace{0.5cm}

	\textbf{You'll then be given a choice}
	"Enter a file which to save the key (/home/You/.ssh/id\_ALGORITHM):"	
	
	\vspace{0.5cm}
	\textbf{Option 1.} \textit{Name the key}
	\begin{itemize}
		\item Enter your chosen key name and press enter
	\end{itemize}

	\textbf{Option 2.} \textit{Press enter}
	\begin{itemize}
		\item Should only choose this option if you will only ever have one key
	\end{itemize}

	\begin{block}{}
		\textbf{Congrats you have created your first SSH Key!!}
	\end{block}

\end{frame}


% adding an ssh key to ssh agent 
\begin{frame}
	\frametitle{\textbf{Adding an SSH Key to the SSH-Agent}}

	\textbf{Step 1.} \textit{Start ssh-agent in the background}
	\begin{itemize}
		\item eval "\$(ssh-agent -s)"
	\end{itemize}

	\vspace{0.5cm}
	\textbf{Step 2.} \textit{Add your ssh key to the ssh-agent}
	\begin{itemize}
		\item ssh-add /.ssh/\(\prec yourKeyName \succ\)
	\end{itemize}

	\vspace{0.5cm}
	\textit{If you didn't name your key use this instead}
	\begin{itemize}
		\item ssh-add /.ssh/id\_ed25519 
	\end{itemize}
\end{frame}

% login into gitlab 
\begin{frame}
	\frametitle{\textbf{Exploring GitLab}}

	\begin{multicols}{2}
	%\begin{figure}[h]
	%		\centering
	%		\includegraphics[width=0.3\textwidth]{img/GitLab_logo.png} 
	%\end{figure}

	\textbf{Open Firefox or Google Chrome}
	\begin{itemize}
		\item https://git.cs.usask.ca/users/sign\_in
	\end{itemize}

	\vspace{0.5cm}	
	\textbf{Username}
	\begin{itemize}
		\item \textit{Your NSID example: abc123}
	\end{itemize}

	\vspace{0.2cm}
	\textbf{Password}
	\begin{itemize}
		\item \textit{Your Canvas Password}
	\end{itemize}

	\begin{figure}[h]
			\includegraphics[width=0.5\textwidth]{img/homer.png} 
	\end{figure}
\end{multicols}
\end{frame}

% adding our key to gitlab 
\begin{frame}
	\frametitle{\textbf{Adding our key to GitLab}}

	\textbf{Step 1.}
	\begin{itemize}
		\item \textit{Click on edit profile}
	\end{itemize}

	\textbf{Step 2.}
	\begin{itemize}
		\item \textit{Click on SSH Keys}
	\end{itemize}

	\textbf{Step 3.}
	\begin{itemize}
		\item \textit{Click on add new key}
	\end{itemize}

	\textbf{Step 4.}
	\begin{itemize}
		\item \textit{Paste public key into key textbox}
	\end{itemize}

	\textbf{Step 5.}
	\begin{itemize}
		\item \textit{Give your key a name}
	\end{itemize}

	\textbf{Step 6.}
	\begin{itemize}
		\item \textit{Click add key}
	\end{itemize}

\end{frame}

% Testing our Connection 
\begin{frame}
	\frametitle{\textbf{Testing our connection}}

	\begin{multicols}{2}
	\textbf{Step 1.}
	\begin{itemize}
		\item git -T git@git.cs.usask.ca 
	\end{itemize}

	\vspace{0.5cm}
	\textbf{Step 2.}
	\begin{itemize}
		\item \textit{Receive a Welcome Message}
	\end{itemize}

	\vspace{0.9cm}
	\begin{block}{\textbf{Troubleshooting Steps}}
		\begin{itemize}
			\item \textit{Re-check you copied the correct public key}
			\item \textit{Re-chck your test command is correct}
		\end{itemize}
	\end{block}
	
			\includegraphics[width=0.5\textwidth]{img/standback.jpeg}
	\end{multicols}
\end{frame}

% important Git Commands 
\begin{frame}
	\frametitle{\textbf{Important Git Commands}}

	\textbf{git clone}
	\begin{itemize}
		\item \textit{Creates a copy of an existing repository}
	\end{itemize}
	\vspace{0.25cm}

	\textbf{git fetch}
	\begin{itemize}
		\item \textit{Updates local references from remote branch but dont merge changes}
	\end{itemize}
	\vspace{0.25cm}

	\textbf{git pull}
	\begin{itemize}
		\item \textit{Fetches changes from remote repository and automatically merges them into your current branch}
	\end{itemize}
	\vspace{0.25cm}

	\textbf{git push}
	\begin{itemize}
		\item \textit{Uploads changes to remote repository} 
	\end{itemize}
	\vspace{0.25cm}

	\textbf{git log}
	\begin{itemize}
		\item \textit{Displays the commit history}
	\end{itemize}

\end{frame}

% what is git branching?
\begin{frame}
	\frametitle{\textbf{What is Git Branching?}}

	\begin{itemize}
		\item Allows you make an isolated copy of your work 
		\item Protects yourself from messing up your final draft 
	\end{itemize}
	\vspace{0.25cm}

	\begin{figure}[h]
			\centering
			\includegraphics[width=0.5\textwidth]{img/broken.jpeg} 
	\end{figure}

\end{frame}

% Making a git branch 
\begin{frame}
	\frametitle{\textbf{Making a Git Branch}}
		

	\textbf{git checkout -b \(\prec branchName \succ\)}
	\begin{itemize}
		\item \textit{Creates a new branch}
	\end{itemize}
	\vspace{0.25cm}

	\textbf{git checkout \(\prec branchName \succ\)}
	\begin{itemize}
		\item \textit{Shows all your branches}
	\end{itemize}
	\vspace{0.25cm}

	\textbf{git checkout -d \(\prec branchName \succ\)}
	\begin{itemize}
		\item \textit{Deletes your specified branch}
	\end{itemize}
	\vspace{0.25cm}

	\textbf{git merge \(\prec branchName \succ\)}
	\begin{itemize}
		\item \textit{Integrate changes from a branch into another branch}
	\end{itemize}
	
\end{frame}

% rolling back in git 
\begin{frame}
	\frametitle{\textbf{What if we need to rollback?}}
	
	\textbf{git reset -soft \(\prec commit\_id \succ\)}
	\begin{itemize}
		\item \textit{Resets to a specific commit, keep changes and staged documents from commits made after the commit}
	\end{itemize}
	\vspace{0.25cm}

	\textbf{git reset -hard \(\prec commit\_id \succ\)}
	\begin{itemize}
		\item \textit{Resets to a specific commit, discards any changes and commits made after the commit}
	\end{itemize}

\end{frame}
\end{document}
